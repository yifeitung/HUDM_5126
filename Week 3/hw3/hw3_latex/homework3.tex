\documentclass{article}
% packages
\usepackage[margin=1 in]{geometry}
\usepackage{indentfirst}
\usepackage{titlesec}
\usepackage{enumitem}

%math package
\usepackage{amsmath}
\usepackage{mathtools}

% Use Links
\usepackage{xcolor}
\usepackage[colorlinks=true ,citecolor=blue, urlcolor=blue,urlbordercolor={1 0 0}]{hyperref} 


\title{HUDM 5126 Linear Models and Regression Analysis Homework 3\footnote{This homework is written in \LaTeX.}}
\author{Yifei Dong}
\date{Sep 16 2020}

\begin{document}
\maketitle

\section{Grade Point Average}
The question is adapted from Q3.3 Refer to Grade Point Average data in Problem 1.19.
\begin{enumerate}[label=(\alph*)]
\item Prepare a boxplot of the ACT scores ($X$ variable). Ate there any noteworthy features on the plot?

\item Prepare a histogram of the residuals. What information does the plot provide?

\item Plot the residuals against the fitted values $\widehat{Y}$. What are your findings about departures from the regression assumptions?

\item Prepare a normal probability plot of the residuals. Test the reasonableness of the normality assumption with the KS test using $\alpha=0.05$. What do you conclude?

\item Conduct the BP test to determine if the variance varies with the level of X. Use $\alpha=0.05$. State your conclusion. Does your conclusion support your preliminary findings in part c)?

\end{enumerate}

\section{Per capita earnings}
A sociologist employed linear regression model (2.1) to relate per capita earnings ($Y$) to average number of years of schooling ($X$) for 12 cities. The fitted values $\widehat{Y_{i}}$ and the semistudentized residuals $e^*_{i}$ follow.

\begin{enumerate}[label=(\alph*)]

\item Plot the semistudentized residuals against the fitted values. What does the plot suggest?

\item How many semistudentized residuals are outside $\pm 1$ standard deviation? Approximately how many would you expect to see if the normal error model is appropriate?

\end{enumerate}

\section{R and RStudio}

Read the data frame from the file HW3.RData by double-clicking on it and opening with RStudio.

\end{document}